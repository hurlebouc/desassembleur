% !TEX root =  ../main_final.tex

\noind Parmi la multitude de probl�mes que l'on rencontre avec le d�veloppement de l'informatique � tous les niveaux; il y a ceux qui trouvent leurs solutions dans l'obfuscation et d'autre part, il y a ceux o� c'est l'obfuscation de programme qui est � l'origine du probl�me.
Et ces probl�mes ne sont pas de moindre importance :\\
\begin{itemize}
\item \textbf{La propri�t� intellectuelle} ou, comment s'assurer que les logiciels que l'on distribue ne soient pas compr�hensibles par les concurrents.
\item \textbf{La lutte contre la contrefa�on} � l'aide de bouts de codes � la fois essentielle au fonctionnement et identifiant le logiciel tout en �tant difficile � localiser dans l'ensemble du code.
\item \textbf{La lutte contre les malwares} qui se complexifient et deviennent de plus en plus difficiles � d�tecter avec l'arriv�e de virus m�tamorphiques.\\
\end{itemize}

\noind Autant de champs d'applications qui ont de pr�s ou de loin pour sujet d'�tudes l'obfuscation. C'est pour cela que dans ce projet de deuxi�me ann�e nous allons essayer de ramener l'analyse d'un ex�cutable � celle d'une repr�sentation normalis�e pouvant s'affranchir des probl�mes soulev�s par l'obfuscation.
De plus, pour connaitre l'effet d'un programme sans en effectuer l'ex�cution \footnote{pour pouvoir effectuer l'analyse sur des programmes malveillants par exemple} il faut connaitre l'�tat de l'ordinateur � chaque �tape de la suite d'instructions. C'est pour cela que l'on va essayer �galement de simuler cette ex�cution.
Nous avons pour cela besoin dans un premier temps de comprendre le fonctionnement des techniques d'obfuscations et les difficult�s inh�rentes � la d�compilation. Une fois que l'on se sera dot� d'outils pour analyser les codes machines, nous pourrons travailler sur la repr�sentation logique du fonctionnement d'un programme. L'ensemble des informations disponibles par d�compilation va aussi nous permettre d'�tudier ses effets mais aussi de r�soudre des probl�mes li�s � l'obfuscation.

